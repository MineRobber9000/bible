\heading{2}{xxxx~--- the people have fallen away from the Lord through transgression~--- they have forsaken God (the wife of their youth\ed{It's difficult to say if this chapter is literal or figurative. However, it's always best to assume that it's symbolic and find the meaning there, and then to figure out if it's literal. If this is purely symbolic, Israel is being called to repentance for forsaking the Lord in their youth and serving other gods. If this is literal, Israelites dealt unfaithfully with their covenant wives, divorced them, and married foreign women. Either way, the message is of faithfulness to spouse and God.}\ed{In ancient Israel there are three elements of marriage: contract, consummation, and celebration. The contract is between the parents of the prospective groom and the prospective bride. Getting out of an engagement could be a very sticky affair. This was tradition from the period before Moses and continued through the end of the monarchic period.}) for foreign gods~--- they call good evil and evil good}\ed{\S~--- This has been a really spiritually powerful chapter for me to translate.}\ed{``Malachi~2 is difficult because it doesn't make much sense.'' ---Professor Ricks, 2015-02-11}

\begin{inparaenum}
    \pvac{\vn{2:1} Now, to you}{is this commandment,}{O priests:}%%
    
    \verse{2:2} ``If you don't listen, if you don't take to heart,\lit{lay it to heart}\alt{pay attention} to give glory to My name,'' says the \textsc{Lord} of Hosts, ``I will send the curse\ed{xxxx Which is?} on you. I will curse your blessings. I have also cursed it because you are not taking it\understood\ to heart.%%
    
    \pvbb{\vn{2:3} I will rebuke\footnotemark\footnotemark\ your posterity}{I've spread dung on your faces,\footnotemark}%%
    \fntalt{reproach}%%
    \fntca{prb 1 \Hebrew{גֹדֵעַ} cf \septuagint\ \Greek{ἀφορίζω} = \Hebrew{גרע}}{probably one has ``cut off,'' compare the Septuagint which has ``excommunicate'' = ``shave/diminish/take away''}%%
    \fnthalot{\textbf{contents of the stomach} (usually of ruminant [even-toed, hoofed mammals that chew the cud regurgitated from their first stomach, comprising the cattle, sheep, antelopes, deer, giraffes, and their relatives] animals); (other: \textbf{dung})}%% Definition taken from Google's "define" command for ruminant, ungulate, and rumen.
    
    \pvcb{the dung of your festivals,\footnotemark}{and it shall take you away with it.}%%
    \fntca{gl, dl}{glossed, to be deleted [this hemistich]}%%
    
    \pvbc{\vn{2:4} You will know}{that\footnotemark\ I have sent you}{this commandment}%%
    \fntca{\septuagint\ + \Greek{ἐγώ}}{the Septuagint adds ``I'' [understood in Hebrew]}%%
    
    \pvcb{so My covenant will be with Levi,''}{says the \textsc{Lord} of Hosts.}%%
    
    \pvbb{\vn{2:5} }{}%%
    
    \pvcb{}{}%%
    
    \pvbb{\vn{2:6} }{}%%
    
    \pvcb{}{}%%
    
    \pvcb{\vn{2:7} }{}%%
    
    \pvca{}%%
    
    \pvbb{\vn{2:8} }{}%%
    
    \pvcb{}{}%%
    
    \pvbb{\vn{2:9} }{}%%
    
    \pvcb{}{}%% double quote needs to be initiated somewhere here or above (or at the beginning of verse 10)
    
    \pvab{\vn{2:10} Do we not all have one father?}{Hasn't one God made us?}%%
    
    \pvbb{Why is a man betrayed by\footnotemark\ his brother}{by profaning the covenant of our fathers?}%%
    \fntalt{deal faithlessly against}%%
    
    \pvac{\vn{2:11} Judah has dealt faithlessly.}{An abomination has been carried out\footnotemark}{against Israel and Jerusalem\footnotemark}%%
    \fntalt{made in}%%
    \fntca{prb dl, var lect}{probably deleted [or] a variant reading [so it would be either ``against Israel'' or ``against Jerusalem,'' not both]}%%
    
    \pvca{\footnotemark because Judah has profaned the holy thing of the \textsc{Lord} in that He's loved and married the daughter of a foreign god.\footnotemark}%%
    \fntca{prb add}{probably added [referring to the entirety from this mark through verse~12]}%%
    \fntie{a foreigner, not of their faith}%%
    
    \pvba{\vn{2:12} The \textsc{Lord} will cut off from the tent of Jacob the man who does this, \Hebrew{עֵר}\footnotemark\footnotemark\ and answers,}%%
    \fntca{prp \Hebrew{עֵד} cf Ms \septuagint\ (\Greek{ἕως} = \Hebrew{עַד})}{it's been proposed to be ``witness/testimony.'' And compare a manuscript of the Septuagint which has ``until''}%%
    \fnted{This word is utterly perplexing. The critical apparatus helps shed some light, but not much.}%%
    
    \pvca{even he who offers a sacrifice to the \textsc{Lord} of Hosts.}%%
    
    \pvaa{\vn{2:13} Here's another thing you should do:\footnotemark\footnotemark}%%
    \fntlit{You do this a second time:}%%
    \fntca{prb add cf 11/12\super{c--c}, al dl sol \Hebrew{שׁנית}}{probably added, compare verses~11 and~12; others add only ``twice''}%%
    
    \pvbc{you cover, with tears,}{the throne of the \textsc{Lord},}{with weeping and groaning,}%%
    
    \pvbb{He no longer regards your offering\footnotemark}{or to receive\footnotemark\ with favor from your hand.}%%
    \fntlit{from where more face of the sacrifice}%%
    \fntalt{take}%%
    
    \pvac{\vn{2:14} You've asked, `Why?'}{Because the \textsc{Lord} has witnessed}{between you and the wife of your youth\footnotemark}%%
    \fnted{Is ``the wife of your youth'' referring to a false god that was served in one's younger days?}%%
    
    \pvbb{that you've been unfaithful with her.}{Yet she is your companions, the wife of your covenant!}%%
    
    \pvaa{\vn{2:15} \footnotemark He didn't make one.\footnotemark\ The remainder of the Spirit was his. What's the one? He is seeking the posterity}%%
    \fntca{prb add}{probably added [this line and the next line]}%%
    \fnted{This could also be rendered, although a few too many gaps are being filled in for me to be comfortable with this, ``Did not one [God] make [us]?''}
    
    \pvca{of\footnotemark\ God. Be careful with your spirit.}%%
    \fntalt{from}%%
    
    \pvea{\footnotemark\footnotemark Don't treat the wife of your youth unfaithfully}%%
    \fntca{exc hemist praecedens (vel compl)?}{has one (or have several) preceding hemistichs been dropped out?}%%
    \fnted{The typesetting here is strange indeed: the previous line is a half stich, then this line starts on a new line but aligned with the preceding stich.}%%
    
    \pvac{\vn{2:16} because I hate divorce,''\footnotemark\footnotemark}{says the \textsc{Lord},}{the God of Israel.\footnotemark}%%
    \fntlit{sending away,''}%%
    \fnted{Divorce, while ``hated'' by God was sometimes necessary, as was the case with Ezra and his foreign wife.}%%
    \fntca{add?}{added? [``says the \textsc{Lord}, the God of Israel'']}%%
    
    \pvbb{``He will cover his garments with violence,''}{says the \textsc{Lord} of Hosts.}%%
    
    \pvca{``Take care of yourselves\footnotemark\ and don't deal unfaithfully.\footnotemark\footnotemark}%%
    \fntlit{``Be careful with your spirit, ``Take care within your spirit}%%
    \fntie{with the wife of your covenant.}%%
    \fntca{prb add cf 15b$\alpha$}{[this line] probably added, compare verse~15}%%
    
    \pvab{\vn{2:17} You've wearied the \textsc{Lord} with your words,}{yet you say, `How\footnotemark\ have we wearied Him?'\footnotemark}%%
    \fntlit{In what}%%
    \fntalt{You?'}%%
    
    \pvbb{In your saying, `Everyone who does evil}{is good in the eyes of the \textsc{Lord}~---}%%
    
    \pvbb{He takes pleasure in them!'}{Or in saying,\footnotemark\ `Where is the God of judgment?'\thinspace''\footnotemark}%%
    \fnted{repeated}%%
    \fnted{This whole idea here of ``sin vigorously because God will forgive'' is wearing to the Lord.}%%
\end{inparaenum}
