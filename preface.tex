\chapter*{Preface}
\section*{The Bible}
The Bible is the word of God so far as it is translated correctly. Personally, the belief that the Bible is infallible shows a clear lack of understanding and education since the Bible is rife with poor translations, mistranslations, and even typos. However, having the Bible in as good of condition as we have it today is a miracle~--- one for which I thank the Lord.

\section*{Translation philosophy}
Growing up with the King James Version of the Holy Bible was a two-edged sword: on one hand, it's a beautifully written and well-accepted version; on the other hand, it's a poetic translation. Personally, non-idiomatic translations show a lack of understanding on the translator's part as to how language works. Poetic translations are difficult to render, but read beautifully; however, they are non-intuitive and therefore not properly suited for most audiences. Therefore, this translation is a rather idiomatic translation with few liberties taken.

\section*{The Tetragrammaton}
The Tetragrammaton (lit., a word having four letters) is the holy name of God, written \Hebrew{יהוה}. In Orthodox Hebrew culture it is unlawful for this word to be uttered by man but once a year by the High Priest on the Day of Atonement in the Holy of Holies. Traditionally, the Tetragrammaton is rendered ``the \textsc{Lord}'' or \textsc{God} (in small caps). This tradition has been adhered to in this edition except in the case of \Hebrew{יְהֹוָה אֲדֹנָי}\footnote{Ketiv. Qere ``adonai elohim.''} where it is usually rendered as ``the Lord \textsc{God}.''\footnote{To avoid rendering it as ``the Lord \textsc{Lord}.''} See further in Appendix~\ref{app:names-of-the-lord}.

\section*{Textual basis}
This text was translated from the \emph{Biblia Hebraica Stuttgartensia}. Inspiration for this translation was taken from the Darby English Bible and Young's Literal Translation. The lexicons used were \emph{The Brown-Driver-Briggs Hebrew and English Lexicon}, Holladay's \emph{A Concise Hebrew and Aramaic Lexicon of the Old Testament}, and Koehler and Baumgartner's \emph{Hebrew and Aramaic Lexicon of the Old Testament}.

\section*{Footnotes and appendix}
Footnotes are used to show alternate renderings and to provide historical, symbolical, and other, expository notes. An appendix appears in the back of the book and contains notes too long for inclusion in footnotes.

\section*{Abbreviations}
alt. = alternatively

lit. = literally % Do this as a table maybe? Check how the BHS does it.

pl. = plural

BH = Biblical Hebrew

Fr. = French

Aram. = Aramaic

Sp. = Spanish

KB = Koehler and Baumgartner's \textit{Hebrew and Aramaic Lexicon of the Old Testament}

Davidson = \textit{The Analytical Hebrew and Chaldee Lexicon} by Benjamin Davidson
