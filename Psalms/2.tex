\heading{2}{A kingship and temple psalm~--- the nations revolt against the Lord~--- He retaliates and anoints His own king~--- blessed are they who trust in the Lord}

\begin{inparaenum}
    \pvbb{\vn{2:1} Why are the nations in unrest?\footnotemark}{Why do the people meditate on vanity?}%%
    \fntalt{agitated [to the point of shaking]?}%%
    
    \pvbb{\vn{2:2} The kings of the earth establish themselves}{and the princes are in cahoots}%%
    
    \pvca{against the \textsc{Lord} and His anointed.\footnotemark}%%
    \fnted{Not necessarily the Anointed, but any of the Lord's servants.}%%
    
    \pvbb{\vn{2:3} Let's tear away their fetters}{and cast off their cords from us.}%%
    
    \pvcb{\vn{2:4} He who dwells in heaven\footnotemark\ shall laugh.}{The Lord\footnotemark\ shall have them in scorn.}%%
    \fntie{deity}%%
    \fntca{\fragheb\ mlt Mss \Hebrew{יהוה}; \septuagint\peshitta\ pr cop}{manuscripts in the Cairo geniza as well as multiple medieval manuscripts have the Tetragrammaton; the Septuagint and the Peshitta put it before the copula [so, right where it is in translation]}%%
    
    \pvbb{\vn{2:5} He will then speak to them in His anger~---}{in His fierce fury\footnotemark\ He will strike them.}%%
    \fnthalot{\Hebrew{חרוֺן} is ``anger (only of God)''}%%
    
    \pvbb{\vn{2:6} I have anointed my king\footnotemark}{on Zion, my holy mountain.\footnotemark}%%
    \fntca{\septuagint\ suff 3 sg}{the Septuagint has a third singular pronominal suffix [i.e., ``his king'']}%%
    \fntca{\septuagint\ suff 3 sg}{the Septuagint has a third singular pronominal suffix [i.e., ``his holy mountain'']}%%
    
    \pvca{\vn{2:7} Let me relate the decree\footnotemark\ the \textsc{Lord}}%%
    \fntca{\peshitta\ + suff 1 sg}{the Peshitta adds a first singular pronominal suffix [i.e., ``my decree:'']}%%
    
    \pvcb{has said to me: ``You are my son.}{Today I have begotten you.}%%
    
    \pvbc{\vn{2:8} Ask of me}{that I may give nations as your inheritance}{and the ends of the earth as your possession.\footnotemark}%%
    \fnted{wonderful example of reverse parallelism}%%
    
    \pvbb{\vn{2:9} You shall break them with an iron sceptre;}{you shall break them in pieces like a potter's utensil.}%%
    
    \pvcb{\vn{2:10} Now, O kings, be prudent.\footnotemark}{You judges of the earth, listen to reason.\footnotemark}%%
    \fntalt{wise}%%
    \fntlit{(niphal imperative) be disciplined}%%
    
    \pvcb{\vn{2:11} Serve the \textsc{Lord} with fear\footnotemark}{and rejoice with trembling. \vn{2:12} Kiss the Son\footnotemark}%%
    \fntca{Ms \Hebrew{בְּשִׂמְחָה}}{a medieval manuscript has ``with joy/gladness''}%%
    \fnted{The \textit{Jewish Study Bible} renders this as ``Pay homage in good faith.''}%%
    
    \pvcb{lest He be angered and you lose your way,}{even though His anger only burns a little.}%%
    
    \pvca{Blessed are all they\footnotemark\ who trust in Him.}%%
    \fntlit{those}%%
\end{inparaenum}
