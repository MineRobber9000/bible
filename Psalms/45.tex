\heading{45}{xxxx}

\begin{inparaenum}
    \verse{45:1} To the music director:\halot{in title at the beginning of a psalm (55~times)\dots\ or at the end of \haref{Hab}{3}{19}; uncertain meaning; traditionally ``for the director of the music,'' Septuagint substantive \Greek{εἰς τὸ τέλος} ($\rightarrow$\thinspace\Hebrew{נֶצַח}), Targum infinitive \Hebrew{לְשַׁבָּחָא} in glorification} \textit{On the Lilies}.\halot{in the titles of Psalms: \dots no certain meaning for this has yet been found,\dots\ suggestions include\dots: $\alpha$) \dots an instruction concerning the tune; $\beta$) \dots either flowers that were placed on or in fron of the Ark (perhaps as a gift of allegiance), or, if \Hebrew{עֵדוּת} means offering) ``the lilies must have been offered as flowers which were to be used in some way to obtain a divine answer to a question that had been presented to a deity'';\dots\ $\gamma$)\dots the title indicated that it was originally a love-song which was subsequently used with wider relevance; $\delta$) Glasser\dots\ takes {\Hebrew{שַׁנִּים}\hspace*{0em}\Hebrew{(וֺ)}\hspace*{0em}\Hebrew{שֹׁ}} etc.\ from Akkadian \textit{\v su\v s\v su} ``one-sixth'', and here it would mean ``a six-stringed instrument''\dots. Deciding which of these suggestions is best must be left open, but the fourth one ($\delta$) has least to support it. $\dagger$ [every Biblical reference quoted]} By the sons of Korah.\ed{Any relation to the infamous Korah of Numbers~16?} An instruction\footnote{xxxx}~--- a song of love.\halot{(for \Hebrew{יְדִידוֺת}) \textbf{love} \haref{Ps}{45}{1}. $\dagger$}\ed{Although plural, \Hebrew{יְדִידֹת} means ``love,'' not ``love(r)s,'' and is undisputed (as given by the $\dagger$).}\ca{\septuagint\symmachus\ sg, 1 c pc Mss \Hebrew{דוּת}\hspace*{0em}-- vel \Hebrew{דֻת}\hspace*{0em}-- cf \aquila\ \Greek{προσφιλίας}}{The Septuagint and Symmachus' Greek translation of the Old Testament give this in singular, one when a few medieval manuscripts give different endings (with a \textit{waw} or a \textit{qubuts}), compare Aquila ``xxxx''}%%
    
    \pvab{\vn{45:2} My heart is stirred up with\footnotemark\ a good thing.}{I will tell what I've done to the king.}%%
    \fntalt{by}%%
    
    \pvca{My tongue is a stylus of an experiences\footnotemark\ writer.}%%
    \fntalt{skilled}%%
    
    \pvab{\vn{45:3} You are more beautiful\footnotemark\ than the sons of man.}{Favor\footnotemark\ is poured out into your lips.}%%
    \fntca{xxxx}{}%%
    \fntalt{Grace}%%
    
    \pvca{Therefore, God\footnotemark\ has blessed you forever.}%%
    \fntca{1 \Hebrew{יהוה}}{one has the Tetragrammaton}%%
    
    \pvab{\vn{45:4} Gird your sword on the\footnotemark\ upper thigh, O mighty one.}{Your majesty\footnotemark\ and your grandeur.\footnotemark \vn{45:5} Your grandeur be successful.\footnotemark\footnotemark}%%
    \fntca{\septuagint\symmachus\peshitta\targum\ + suff 2 sg}{The Septuagint, Symmachus' Greek translation of the Old Testament, the Peshitta, and the Targum all add a second singular pronominal suffix [thus, ``your (upper) thigh'']}%%
    \fntalt{splendor, height}%%
    \fntalt{splendor, glory}%%
    \fntca{crrp, 1 \Hebrew{חֲלָצֶיךָ} \Hebrew{הֲדֹר}}{corruption, one has ``xxxx''}%%
    \fntca{huc tr \Hebrew{׃}}{hither transpose a \textit{sof pasuq} (symbol for end of verse)}%%
    
    \pvca{Ride on the word of truth and humility\footnotemark\ of righteousness.\footnotemark}%%
    \fnthalot{(one who understands himself to be) \textbf{low}, \textbf{humble}, \textbf{gentle} (before God)}%%
    \fntca{Ms `\Hebrew{צ}~\Hebrew{וַת}\hspace*{0em}-- cf \aquila\peshitta, \septuagint(\targum\ Hier) \Greek{καὶ πραύτητος καὶ δικαιοσύνης}; 1 frt sol \Hebrew{וְצֶדֶק}, prp \Hebrew{הַצֶּדֶק} \Hebrew{וְיַעַן}}{a medieval manuscript has a variant ending (\Hebrew{צֶדֶק} \Hebrew{וְעַנְוָת} instead of \Hebrew{וְעַנְוָה־צֶדֶק}), compare Aquila and the Peshitta. The Septuagint (and the Targum and Hieronymus' Greek translation of the Old Testament) have ``xxxx''. One perhaps only has ``and righteousness.'' It has been proposed to be ``and answers righteousness''}%%
    
    \pvbb{Your right hand will teach you fearful things.\footnotemark}{\vn{45:6} \footnotemark Your arrows are sharp.\footnotemark\ People fall under you}%%
    \fnthalot{king's fearful deeds}%%
    \fntca{huc tr $^\text{c--c}$}{hither transpose ``People fall under you.''}%%
    \fntca{\septuagint\ + \Greek{δυνατέ}, ins \Hebrew{הַגִּבּוֺר}}{The Septuagint adds ``xxxx,'' insert ``mighty''}%%
    
    \pvca{in the heart of the king's enemies.}%%
    
    \pvab{\vn{45:7} Your throne, O God, is forever\footnotemark\ and always.}{A staff of order\footnotemark\ is a staff of Your kingdom.}%%
    \fntca{pc Mss et var sec Odonem ´\Hebrew{לְע} cf \septuagint\aquila\theodotion}{A few medieval manuscripts and variant readings according to Odonem add ``to/for,'' compare the Septuagint, Aquila, and Theodotion's Greek translation of the Old Testament}%%
    % Odonem?: http://www.vanhamel.nl/wiki/Liber_de_anima_ad_Odonem_Bellovacensem
    \fntalt{regulation, uprightness, justice}%%
    
    \pvba{\vn{45:8} You love righteousness and hate\footnotemark\ wickedness.}%%
    \fntalt{are unable to put up with}%%
    
    \pvbb{Therefore, God\footnotemark\ (your God) has anointed you}{with oil of joy\footnotemark\ above your companions.}%%
    \fntca{Ms \targum\ \Hebrew{יהוה}}{a medieval manuscript and the Targum use the Tetragrammaton}%%
    \fntalt{exultation}%%
    
    \pvba{\vn{45:9} }%%
    
    \pvbb{}{\vn{45:10} }%%
    
    \pvca{}%%
    
    \pvab{\vn{45:11} }{}%%
    
    \pvba{\vn{45:12} }%%
    
    \pvbb{ \vn{45:13} }{}%%
    
    \pvba{\vn{45:14} }%%
    
    \pvbb{ \vn{45:15} }{}%%
    
    \pvba{\vn{45:16} }%%
    
    \pvaa{}%%
    
    \pvab{\vn{45:17} }{}%%
    
    \pvab{\vn{45:18} }{}%%
\end{inparaenum}
