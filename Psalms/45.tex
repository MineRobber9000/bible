\heading{45}{Either an erotic psalm (unlikely\ed{``Trust me, this is not erotic.'' ---Professor Ricks, 2015-01-28}) or a psalm of marriage, a wedding hymn, and symbolically theological~--- the groom is more beautiful than the sons of men~--- he shall gird himself in glory and majesty~--- God's throne is forever and always~--- the groom is anointed~--- his name will always be remembered}

{\noindent\textit{\small Note: The king (groom) is being addressed in verses~1--10, the bride (princess) in verses~11--16, and the king again in verses~17--18.}}

\begin{inparaenum}
    {\noindent\verse{45:1} To the music director:\halot{in title at the beginning of a psalm (55~times)\dots\ or at the end of \haref{Hab}{3}{19}; uncertain meaning; traditionally ``for the director of the music,'' Septuagint substantive \Greek{εἰς τὸ τέλος} ($\rightarrow$\thinspace\Hebrew{נֶצַח}), Targum infinitive \Hebrew{לְשַׁבָּחָא} in glorification} \textit{On the Lilies}.\halot{in the titles of Psalms: \dots no certain meaning for this has yet been found,\dots\ suggestions include\dots: $\alpha$) \dots an instruction concerning the tune; $\beta$) \dots either flowers that were placed on or in fron of the Ark (perhaps as a gift of allegiance), or, if \Hebrew{עֵדוּת} means offering) ``the lilies must have been offered as flowers which were to be used in some way to obtain a divine answer to a question that had been presented to a deity'';\dots\ $\gamma$)\dots the title indicated that it was originally a love-song which was subsequently used with wider relevance; $\delta$) Glasser\dots\ takes {\Hebrew{שַׁנִּים}\hspace*{0em}\Hebrew{(וֺ)}\hspace*{0em}\Hebrew{שֹׁ}} etc.\ from Akkadian \textit{\v su\v s\v su} ``one-sixth'', and here it would mean ``a six-stringed instrument''\dots. Deciding which of these suggestions is best must be left open, but the fourth one ($\delta$) has least to support it. $\dagger$ [every Biblical reference quoted]} For\footnotemark\ the sons of Korah.\ed{Could easily be a different Korah than the infamous one in Numbers~16.} An instruction~--- a song of love.\halot{(for \Hebrew{יְדִידוֺת}) \textbf{love} \haref{Ps}{45}{1}. $\dagger$}\ed{Although plural, \Hebrew{יְדִידֹת} means ``love,'' not ``love(r)s,'' and is undisputed (as given by the $\dagger$).}\ca{\septuagint\symmachus\ sg, 1 c pc Mss \Hebrew{דוּת}\hspace*{0em}-- vel \Hebrew{דֻת}\hspace*{0em}-- cf \aquila\ \Greek{προσφιλίας}}{The Septuagint and Symmachus' Greek translation of the Old Testament give this in singular, one when a few manuscripts give different endings (with a \textit{waw} or a \textit{qubuts}), compare Aquila ``marriage''}}%%
    \fntalt{By, With}%%
    
    \pvab{\vn{45:2} My heart is moved\footnotemark\ by\footnotemark\ a pleasant thing.}{I will say my verses\footnotemark\ to the king.}%%
    \fntalt{stirred up}%%
    \fntalt{with}%%
    \fntalt{what I've done, my doings/deeds/words}%%
    
    \pvca{My tongue is a stylus of an experienced\footnotemark\ scribe.\footnotemark}%%
    \fntalt{skilled, ready}%%
    \fntalt{writer}%%
    
    \pvab{\vn{45:3} You are fairer than the children of men.\footnotemark}{Favor\footnotemark\ is poured out into your lips.}%%
    \fnted{``Shocked, on the other hand, by these revolting fancies, there were many who held that Jesus, in His earth;y features reflected the charm and beauty of David, His great ancestor; and St.\ Jerome and St.\ Augustine preferred to apply to Him the words of Psalm~xlv.\ 2, 3, `Thou art fairer than the children of men'\thinspace'' (\textit{The Life of Christ}, Farrar, quoting Augustine \textit{in Ep.\ Joh.}, tract.\ ix.\ 9).}%%
    \fntalt{Grace}%%
    
    \pvca{Therefore, God\footnotemark\ will blessed you forever.}%%
    \fntca{1 \Hebrew{יהוה}}{one has the Tetragrammaton}%%
    
    \pvab{\vn{45:4} Gird your sword on the\footnotemark\ thigh,\footnotemark\footnotemark\ O hero.}{Gird your majesty\footnotemark\ and your grandeur\footnotemark~--- \vn{45:5} have success in your glory.\footnotemark}%%
    \fntca{\septuagint\symmachus\peshitta\targum\ + suff 2 sg}{The Septuagint, Symmachus' Greek translation of the Old Testament, the Peshitta, and the Targum all add a second singular pronominal suffix [thus, ``\textit{your} thigh'']}%%
    \fntlit{upper thigh}%%
    \fnted{This is the part that could, because this is a wedding hymn, make this erotic. However, this is more probably about enthronement.}%%
    \fntalt{splendor, height}%%
    \fntalt{splendor, glory}%%
    \fntca{huc tr \Hebrew{׃}}{hither transpose a \textit{sof pasuq} (symbol for end of verse)}%%
    
    \pvca{Ride for the sake\footnotemark\ of truth and humility\footnotemark\ and righteousness.}%%
    \fntlit{on the word}%%
    \fnthalot{(one who understands himself to be) \textbf{low}, \textbf{humble}, \textbf{gentle} (before God)}%%
    
    \pvbb{May your right hand teach you awe-inspiring\footnotemark\ things.\footnotemark}{\vn{45:6} \footnotemark Your arrows are sharp.\footnotemark\ People fall beneath you}%%
    \fntalt{amazing, wonderful, terrifying}%%
    \fnthalot{king's fearful deeds}%%
    \fntca{huc tr \super{c--c}}{hither transpose ``People fall under you.''}%%
    \fntca{\septuagint\ + \Greek{δυνατέ}, ins \Hebrew{הַגִּבּוֺר}}{The Septuagint adds ``be powerful,'' insert ``mighty''}%%
    
    \pvca{in the midst\footnotemark\ of the king's enemies.}%%
    \fntlit{heart}%%
    
    \pvab{\vn{45:7} Your throne, O God, is forever\footnotemark\ and\footnotemark\ always.}{A staff of Your kingdom is a staff of order.\footnotemark\footnotemark}%%
    \fntca{pc Mss et var sec Odonem ´\Hebrew{לְע} cf \septuagint\aquila\theodotion}{A few manuscripts and variant readings according to Odo [Bishop of Beauvais] [``-nem'' is a the Latin accusative] add ``to/for,'' compare the Septuagint, Aquila, and Theodotion's Greek translation of the Old Testament}%%
    \fnted{See further in Appendix~\ref{app:waw-voweling}}%%
    \fntalt{righteousness, regulation, uprightness, justice}%%
    \fntlit{A staff of order is a staff of Your kingdom.}%%
    
    \pvba{\vn{45:8} You love righteousness and hate\footnotemark\ wickedness.}%%
    \fntalt{are unable to put up with}%%
    
    \pvbb{Therefore, God\footnotemark\ (your God) has anointed you}{with oil of joy\footnotemark\ more than your friends.\footnotemark}%%
    \fntca{Ms \targum\ \Hebrew{יהוה}}{a manuscript and the Targum use the Tetragrammaton}%%
    \fntalt{exultation}%%
    \fntalt{companions, associates, fellows}%%
    
    \pvba{\vn{45:9} Myrrh\footnotemark\ and aloes,\footnotemark\ \footnotemark cinnamon-flowers,\footnotemark\ all your clothes\footnotemark~---}%%
    \fnthalot{\textbf{myrrh}, resin of \textit{Commiphora abessinica}}%%
    \fnthalot{\textbf{aloes} (aromatic wood), \textit{Aloexyllon Agallochum} \& \textit{Aquilaria Agallocha}}%%
    \fntca{pc Mss Vrs ´\Hebrew{וּק}}{a few manuscripts, all or most manuscripts, have ``and''}%%
    \fnthalot{\textbf{cassia}, \textbf{cinnamon-flowers} (dried for incense)}%%
    \fnted{\dots have been scented with}%%
    
    \pvbb{from ivory palaces, stringed instruments\footnotemark\ have given you joy.\footnotemark}{\vn{45:10} The kings' daughters\footnotemark\ stand firm among your nobles.\footnotemark}%%
    \fnted{``stringed instruments from ivory palaces''?}%%
    \fntalt{made you glad/joyful/rejoice}%%
    \fntca{\peshitta\ \textit{brt mlk'}, 1 \Hebrew{הַמֶּלֶךְ} \Hebrew{בַּת}}{the Peshitta has ``daughters of the king'' and one manuscript has ``a daughter of the king''}%%
    \fnthalot{\textbf{rare}\dots\ \textbf{precious stone}\dots precious (building-)stones\dots \textbf{costly}, \textbf{valuable}\dots \textbf{noble}}%%
    
    \pvca{A queen\footnotemark\footnotemark\ is\footnotemark\ at your right hand in gold from Ophir.}%%
    \fnthalot{traditionally \textbf{queen}\dots\ but suggested `favorite of harem'. $\dagger$}%%
    \fntalt{consort}%%
    \fnted{Some translations give ``stands at,'' but this verb is not present in L.}%%
    
    \pvab{\vn{45:11} Listen, O daughter, and see. Listen.\footnotemark}{Forget your people and your father's house.\footnotemark}%%
    \fntlit{Incline your ear.}%%
    \fnted{When women come from other lands to marry, they give up their now-foreign traditions.}%%
    
    \pvba{\vn{45:12} The king\footnotemark\ will long for\footnotemark\ your beauty because he is your lord!---}%%
    \fntca{prb dl}{probably to be deleted [making it ``he will long for''?]}%%
    \fntalt{crave}%%
    
    \pvbb{worship\footnotemark\ him.\footnotemark\footnotemark\ \vn{45:13} The daughter of Tyre with an offering,}{the rich people will flatter\footnotemark\ your face.\footnotemark}%%
    \fntalt{bow down before}%%
    \fntca{1 frt \Hebrew{לָךְ} \Hebrew{וָה}\hspace*{0em}--- et cj c 13}{one manuscript has ``he shall worship you,'' and it [this thought] connects with verse~13}%%
    \fntca{\septuagint\ 3 pl}{the Septuagint gives this in third person plural [i.e., ``you shall worship them'']}%%
    \fntalt{appease}%%
    \fnted{in reference to her beauty; easy on the eyes}%%
    
    \pvbb{\vn{45:14} The princess\footnotemark\footnotemark\ is glorious within~---}{\footnotemark her clothes are woven\footnotemark}%%
    \fntca{prb dl m cs}{probably deleted because of the meter case}%%
    \fntlit{daughter of the king}%%
    \fnted{works better as a separate stich}%%
    \fnted{the root (\Hebrew{שׁבּץ}) means to ``weave in patterns''}%%
    
    \pvbb{with gold.\footnotemark\ \vn{45:15} \footnotemark She will be brought\footnotemark\ before the king in brightly colored clothes.\footnotemark}{The virgins\footnotemark\ and her friends\footnotemark\ after her shall be brought in to you.\footnotemark}%%
    \fnted{reworked because Hebrew grammar is non-idiomatic for English. This literally says ``are woven with gold / her clothes.''}%%
    \fntca{15.\ 16 crrp, metrum inc}{verses 15 and 16 corrupted, meter uncertain}%%
    \fnthalot{as a bride}%%
    \fntlit{fabric of a variety of colors.}%%
    \fnthalot{\textbf{virgin}:---1.\ mature girl `whom no man has known'}%%
    \fnted{more probably just a ``young woman'' with no particular regard for whether she's a virgin}%%
    \fntca{prb dl}{probably deleted}%%
    \fntca{2 Mss \Hebrew{לָהּ}, \missing\ \septuagint\super{min}\ \peshitta; prp \Hebrew{הֹלְכֹת} (hpgr)}{two manuscripts have ``to her,'' but this is missing from the Septuagint (codices minusculis scripti) and the Peshitta. It has been proposed that there is missing a verb meaning ``to bring in'' (by haplography [the characters present are \Hebrew{לך} and it's been proposed that a second \Hebrew{כ} is missing which would make \Hebrew{הלכת}])}%% xxxx make sense of this haplography
    
    \pvba{\vn{45:16} They are brought in with joy and rejoicing. They enter the palace of the king.\footnotemark}%%
    \fnted{although in the \textsc{bhs} it appears that ``of the king'' is on a separate line, this is just a typesetting problem and it should not be its own stich.}%%
    
    \pvab{\vn{45:17} Instead of your fathers it shall be your sons.\footnotemark}{You shall make\footnotemark\ them\footnotemark\ princes throughout the whole world.}%%
    \fntie{your posterity will succeed your fathers}%%
    \fnted{\Hebrew{מוֹ}\hspace*{0em}--- is the poetic equivalent of \Hebrew{הֶן}\hspace*{0em}--- and \Hebrew{הֶם}\hspace*{0em}---}%%
    \fntalt{appoint}%%
    
    \pvab{\vn{45:18} I will cause your name\footnotemark\ to be remembered\footnotemark\ throughout all generations.\footnotemark}{Therefore, the people shall praise You forever and always.\footnotemark}%%
    \fntie{the king's name}%%
    \fntlit{(cohortative) let me mention your name}%%
    \fntalt{for endless generations.}%%
    \fntca{prb dl alterutrum m cs}{either one or the other [either ``[t]herefore'' or ``and always''] should be deleted because of the meter case}%%
\end{inparaenum}
