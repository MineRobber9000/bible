\heading{15}{Only those who are pure and clean my enter the temple~--- qualifications for cleanliness enumerated}

\begin{inparaenum}
    {\noindent\verse{15:1} A psalm of David:}%%
    
    \pvcb{O Lord, who shall dwell in Your tent?}{Who will reside in Your holy mountain?}%%
    
    \pvbb{\vn{15:2} Those who walk uprightly,\footnotemark}{work righteousness~---}%%
    \fnted{Although \Hebrew{תָּמִים} can mean ``perfectly,'' it is more modest to here render it as ``uprightly.''}%%
    
    \pvcb{who speaks truth from his heart.}{\vn{15:3} Who doesn't slander with his tongue,}%%
    
    \pvcb{do evil to his friends,}{or lift up reproaches on his neighbor.}%%
    
    \pvbb{\vn{15:4} The rejected\footnotemark\footnotemark\ are despised\footnotemark\ in his eyes;}{he honors those who fear the \textsc{Lord}.}%%
    \fntca{crrp}{corruption}%%
    \fnted{This makes enough sense as is that the corruption doesn't seem to make a noticeable difference.}%%
    \fntca{2 Mss `\Hebrew{וְנ}, \septuagint\ \Greek{πονηρευόμενος}, \peshitta\ \textit{mrgzn'} irritator}{two medieval manuscripts have ``and he honors,'' the Septuagint has ``doing evil,'' the Peshitta has ``he who irritates''}%%
    
    \pvcb{If he's sworn to his own inconvenience,\footnotemark\footnotemark}{he doesn't retract.}%%
    \fntalt{hurt}%%
    \fntca{\septuagint(\peshitta) \Greek{τῷ πλησίον αὐτοῦ}, \symmachus\ \Greek{ἐταῖρος εἶναι}}{Septuagint (and Peshitta) ``to his neighbor,'' Symmachus' Greek translation of the Old Testament ``fellow companion''}%%
    
    \pvbb{\vn{15:5} He doesn't put his money to usury,}{doesn't take bribes\footnotemark\ against the innocent.}%%
    \fntalt{rewards}%%
    
    \pvcb{He who does these things}{shall never be shaken.}%%
\end{inparaenum}
