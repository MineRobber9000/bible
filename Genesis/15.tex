\heading{15}{The Lord promises great things to Abram~--- Abram asks the Lord what is being promised, seeing as he has no children~--- }

\begin{inparaenum}
    \verse{15:1} After these things, the word of the \textsc{Lord} came to Abram in a vision, saying, ``Don't be afraid, Abram. I am your refuge.\alt{protection, shield}\ed{This is so incredibly comforting.} Your reward will be very great.''%%
    \verse{15:2} Abram said, ``Lord \textsc{God}, what are you going to give me? I am\lit{go} childless.\ed{This is such a touching exchange and gives us a good look into the emotions and personality of Abram. After all this time and after all the marvelous things that the Lord has given him, when presented with being given more his first thought turns to being childless. For Abram, having a child is his ultimate wish. He's been promised countless things before~--- and has even seen some of those promises fulfilled~--- but his ultimate desire, especially in his older age, is to have a child. Additionally, he's been promised seed beyond measure. His hopes have been lifted this whole time, yet he and Sarai have never had success in having a child. How heartbroken he must be! To be promised things from God and not see them fulfilled. Did he feel empty? Forsaken? Or did he know that God would eventually fulfill that promise? And maybe he even knew that it might only be a promise to be fulfilled after this life. But these things have stayed with him, festered in his mind, to the point that when God promises him a very great reward, he immediately asks how that can be because he is childless.} The [unexplained] son\halot{\Hebrew{מֶשֶׁק}, \sampen$.^\text{M216}$ \textit{m\=a\v saq}, \septuagint\ \Greek{Μασεκ}: \Hebrew{בֵּיתִי} \Hebrew{מֶשֶׁק} \Hebrew{בֶּן}, subsequently glossed with \Hebrew{דַמֶּשֶׁק} \Hebrew{הוּא} \haref{Gn}{15}{2}; unexplained., ? Ug.\ \textit{m\v sq}}\ed{Possibly: steward (\textsc{darby}, \textsc{kjv}), acquired son (\textsc{ylt}), inheritor (\textsc{lsg}); all of these are understood through further context given in verse~3.} is Eliezer of Damascus.''\ca{prb gl aram ad \Hebrew{בן־משׁק}}{probably an Aramaic gloss to \Hebrew{בן־משׁק} (son of [unexplained])}%%
    \verse{15:3} And Abram said, ``You haven't given me posterity, and a son of my house will be my heir.''%%
    \verse{15:4} %%
    \verse{15:5} %%
    \verse{15:6} %%
    \verse{15:7} %%
    \verse{15:8} %%
    \verse{15:9} %%
    \verse{15:10} %%
    \verse{15:11} %%
    \verse{15:12} %%
    \verse{15:13} %%
    \verse{15:14} %%
    \verse{15:15} %%
    \verse{15:16} %%
    \verse{15:17} %%
    \verse{15:18} %%
    \verse{15:19} %%
    \verse{15:20} %%
    \verse{15:21} %%
\end{inparaenum}
