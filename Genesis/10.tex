\heading{10}{Noah's genealogy set forth}

\begin{inparaenum}
    \verse{10:1} These are the generations of Noah's children, Shem, Ham, and Japheth; sons were born to them after the Flood.%%
    \verse{10:2} The sons\ed{Could also be rendered ``children,'' but that language would exclude the possibility of daughters.} of Japheth: Gomer, Magog, Madai, Javan, Tubal, Meshech, Tiras.%%
    \verse{10:3} Gomer's sons: Ashkenaz, Riphath, Togarmah.%%
    \verse{10:4} Javan's sons: Elishah, Tarshish, Kittim, Dodanim.%%
    \verse{10:5} From these branched off\halot{family groups genealogically} the island nations:\halot{the Phoenicians in Is~$23_{2-6}$ (from \Hebrew{אִי יֹשְׁבֵי}). The islands and coasts of the Mediterranean are, for the Old Testament, the extremes of the western world.} by their lands, each by his language, by their family, by their nations.%%
    \verse{10:6} Ham's sons: Cush, Mizraim,\footnote{Hebrew for Egypt} Phut, Canaan.%%
    \verse{10:7} Cush's sons: Seba, Havilah, Sabtah, Raamah, Sabtecha. Raamah's sons: Sheba and Dedan.%%
    \verse{10:8} Nimrod was born to Cush, and he began to be a despot\alt{mighty hunter, manly, vigorous} in the land.%%
    \verse{10:9} He was a hunting despot\halot{gives \Hebrew{גִבּוֺר} as ``Nimrod was a \textbf{despot}'' and \Hebrew{צַיִד} as ``\textbf{hunting}.''} before the \textsc{Lord}; hence\alt{therefore} it is said, ``Like Nimrod, the hunting despot before the \textsc{Lord}.''%%
    \verse{10:10} The beginning of his realm was Babel\alt{Babylon, Persia} and Erech and Accad and Calneh, in the land of Shinar.%%
    \verse{10:11} Assyria stretched out from that land, and he\ie{Nimrod} built Nineveh, Rehoboth-Ir,\ed{The Hebrew (\Hebrew{עִיר רְחֹבֹת}) is said to be a location in \textsc{halot}; however, \Hebrew{רְחֹבֹת} is also the plural of \Hebrew{רְחֹב} meaning ``\textbf{square}, \textbf{plaza}.'' Therefore, this could also mean ``streets (or public square or plaza) of the city'' and refer to the infrastructure of Nineveh and not a different city.} Calah,%%
    \verse{10:12} Resen (between Nineveh and Calah), which is a great city.%%
    \verse{10:13} Mizraim fathered the Ludim, the Anamim, the Lehabim, the Naphtuhim,%%
    \verse{10:14} the Pathrusim, the Casluhim (from whom came the Philistines), and the Caphtorim.%%
    
    \verse{10:15} Canaan fathered his firstborn, Sidon, and Heth,%%
    \verse{10:16} the Jebusite, the Amorite, the Girgashite,%%
    \verse{10:17} the Hivite, the Arkite, the Sinite,%%
    \verse{10:18} the Arvadite, the Zemarite, the Hamathite. Afterwards, the Canaanite tribes\halot{extended family, \textbf{clan} (group in which there is a felt blood-relationship)} were scattered.\alt{spread out, dispersed}%%
    \verse{10:19} %%
    \verse{10:20} %%
    
    \verse{10:21} %%
    \verse{10:22} %%
    \verse{10:23} %%
    \verse{10:24} %%
    \verse{10:25} %% Peleg and the countries being split up probably refers to the nations and not the tectonic plates.
    \verse{10:26} %%
    \verse{10:27} %%
    \verse{10:28} %%
    \verse{10:29} %%
    \verse{10:30} %%
    \verse{10:31} %%
    \verse{10:32} %%
\end{inparaenum}
