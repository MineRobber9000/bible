\heading{4}{Adam and Eve have two sons, Cain and Abel~--- Cain raises crops, Abel tends flocks~--- Cain and Abel both present sacrifices to the Lord, but only Abel's sacrifice is accepted~--- Cain kills Abel~--- the Lord banishes Cain from society~--- xxxx}

\begin{inparaenum}
    \verse{4:1} Adam had intercourse with his wife Eve and she became pregnant and gave birth to Cain. She said, ``I have produced a man with the \textsc{Lord}.''\footnote{Understood ``with His help.''}%%
    \verse{4:2} She continued to birth his brother, Abel. Abel was shepherding his sheep and goats,\heb{צֹאן}{} and Cain was a tiller of the ground.%%
    \verse{4:3} Sometime later, Cain brought in a sacrifice\halot{older passages: offering or sacrifice of homage, allegiance (of either meat or cereal)} to the \textsc{Lord} from the fruit of the ground.%%
    \verse{4:4} Also, Abel brought in the firstborn\halot{of cattle}\ed{\textsc{ylt} is incorrect in stating this as ``female firstlings.'' The noun is feminine, but that does not mean that they are female cattle, merely that the noun is feminine.} of his sheep and goats, and of their fatty pieces.\halot{pieces of fat} The \textsc{Lord} looked with favor on Abel and his sacrifice.%%
    \verse{4:5} But He did not look with favor on Cain and his sacrifice, and Cain became very much indeed\ed{Word for word from \textsc{halot}.} angry and he was downcast.\alt{his face sagged.}%%
    \verse{4:6} The \textsc{Lord} said to Cain, ``Why are you angry? Why are you downcast?%%
    \verse{4:7} If you do well, shall you not be exalted?\footnote{The abridged \textsc{halot} gives: \textbf{exaltation} (?) refer to commentaries. We then turn to the unabridged \textsc{halot} which gives: \textbf{raising, lifting up} a phrase completed with \Hebrew{פָּנִים}, the raising of the face, countenance Gn~$4_7$. It's important to here note that this does not refer to lifting oneself up at the last days, which, beyond being theologically impossible for all but the Savior, is definitionally impossible because the object is \Hebrew{פָּנִים}. However, the first definition (which \textsc{halot} doesn't explicitly say refers to Gen~4\thinspace:\thinspace7) gives: \textbf{elevation, exaltation} (to which one raises oneself), ascent sbj.\ \Hebrew{לִוְיָתָן} Jb~$41_{17}$.} If you don't do well, sin lurks at the entryway\lit{entrance}~--- its desire will be towards you and it will gain dominion over you.''%%
    \verse{4:8} Cain said to his brother Abel, ``Let's go into the field.''\footnote{Footnote~8a in \textsc{bhs} says, ``mlt Mss Edd hic interv; frt ins c \sampen\septuagint\peshitta\vulgate \Hebrew{נֵלְכָה הַשָּׂדֶה} cf \targum$^\text{J\thinspace J\thinspace I\thinspace I}$ (\textit{many editions of Hebrew manuscripts this interval}).'' This is an exasperated way of saying that many manuscripts contain an omission in \masoretic\ which states ``Let's go into the field.'' See further in Tov 2012 p.\ 221.} So they were in the field and Cain rose up against his brother Abel and killed him.%%
    \verse{4:9} The \textsc{Lord} said to Cain, ``Where is your brother Abel?'' He replied, ``I don't know. Do I have to keep watch of my brother?''%%
    \verse{4:10} He said, ``What have you done? The sound of your brother's blood cries out to me from the ground!%%
    \verse{4:11} And now, you shall be cursed from the ground which has opened its mouth to take your brother's blood from your hand.%%
    \verse{4:12} When you till the ground, it will never again\lit{not again} yield its strength to you. You shall be a wanderer, and homeless, in the earth.''%%
    \verse{4:13} Cain said to the \textsc{Lord}, ``My punishment\halot{for guilt} is too great for me to bear.%%
    \verse{4:14} Look, today you've banished me\alt{driven me out} from\lit{from off} the face of the land. From your face I am hid. I will be a wanderer, and homeless, in the earth, and it shall be that all those who find me will kill me.''%%
    \verse{4:15} %%
    \verse{4:16} %%
    
    \verse{4:17} %%
    \verse{4:18} %%
    \verse{4:19} %%
    \verse{4:20} %%
    \verse{4:21} %%
    \verse{4:22} %%
    \verse{4:23} %%
    
    \verse{4:24} %%
    
    \verse{4:25} %%
    \verse{4:26} %%
\end{inparaenum}
