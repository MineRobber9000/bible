\heading{2}{God completes His work and hallows the seventh day~--- xxxx}

\begin{inparaenum}
    \verse{2:1} The heavens, the earth, and their hosts were completed.\footnote{lit., made complete}%%
    \verse{2:2} God completed His work\footnote{Might be closer to ``life's work,'' although that doesn't theologically work in the eternal sense. The Hebrew, \Hebrew{מְלַאכְתּוֺ} (from \Hebrew{מְלַאכָה}), also means \textit{(business) mission}, \textit{business trip}, \textit{business}, \textit{occupation}, \textit{labor}, \textit{enjoyment}, \textit{craft}, \textit{job}, or \textit{task}.} that he'd done by\footnote{lit., on; but this doesn't work theologically.} the seventh day and, on the seventh day, stopped\footnote{From \Hebrew{שׁבת}} all His work that he'd been doing.%%
    \verse{2:3} God blessed the seventh day~--- He sanctified it~--- because on it he stopped all\footnote{What are the theological implications of \textit{all}?} His work that God had created\footnote{HEB \Hebrew{בָּרָא}} to do.\footnote{alt., make, produce}%%
    
    \verse{2:4} These are the origins of the heavens and the earth, when they were created, in the day the \textsc{Lord} God made the earth and the heavens,%%
    
    \verse{2:5} every bush\footnote{alt., shrub} of the field before it was on the earth, and every green plant\footnote{KB: weeds, grass, vegetables, cereals, growing during rainy season, not perennials; seed-producing plants; seed-bearing plants} of the field before they sprouted (because the \textsc{Lord} God had not yet\understood\ let it rain on the earth and there was no man to till and cultivate\footnote{Both of these are possible definitions of \Hebrew{עבד}, but I feel both are needed to properly define what is being said.} the ground.%%
    \verse{2:6} The subterranean fresh-water stream\footnote{Straight from KB. Sounds like existing ground water, possibly flowing. In Job 36\thinspace:\thinspace27 it refers to the \textbf{heavenly stream}.} rose up from the earth and watered the entire surface\footnote{lit., face} of the ground.%%
    \verse{2:7} The \textsc{Lord} God created\footnote{HEB \Hebrew{יָצַר}; KB: older, concrete synonym of \Hebrew{ברא}} man\footnote{alt., mankind; however, this would not be theologically correct.}~--- loose soil\footnote{alt., dry, fine particles of dirt; dust} from the ground~--- and breathed the breath of life into his nostrils: and man became a living soul.%%
    \verse{2:8} The \textsc{Lord} God planted, in the east, a garden in Eden; He there put the man whom He had created.%%
    \verse{2:9} The \textsc{Lord} God made every tree that is excellent\footnote{alt., desirable, precious, beloved} to behold and good for food to\ed{Changed to infinitive to read more idiomatically.} sprout from the ground; the tree of life to sprout\understood\ in the midst of the garden; and the tree of the knowledge of good and evil.\footnote{Understood that this is also sprouting in the garden. Also understood that this is a fragment sentence.}%%
    \verse{2:10} %%
    \verse{2:11} %%
    \verse{2:12} %%
    \verse{2:13} %%
    \verse{2:14} %%
    \verse{2:15} %%
    \verse{2:16} %%
    \verse{2:17} %%
    \verse{2:18} %%
    \verse{2:19} %%
    \verse{2:20} %%
    \verse{2:21} %%
    \verse{2:22} %%
    \verse{2:23} %%
    
    \verse{2:24} %%
    \verse{2:25} %%
\end{inparaenum}
