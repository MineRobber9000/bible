\heading{20}{The Decalogue~--- Israel commanded to bear witness that the Lord has spoken~--- altars of unhewn stone are to be built~--- sacrifices performed thereon}

\begin{enumerate*}[mode=unboxed]
    \verse{20:1} And God spake all these words unto them, saying,%%
    \verse{20:2} \textsc{Preface.} ``I am the \textsc{Lord} your God who brought you out of the land of Egypt, from the house of captivity.\footnote{servitude, bondage, slavery}%%
    \verse{20:3} \textsc{i.}\footnote{There are varied approaches to numbering the commandments. The Philonic tradition is used here.} Never\footnote{The Hebrew negator \Hebrew{לוֹא} is used here. It is used when someone in authority is speaking to an inferior. When Moses speaks, he uses \Hebrew{אַל}~--- a word that is spoken between equals.} shalt thou have other gods besides me.\footnote{lit., my face. The Greek rendering is used herein.}\footnote{In the BHS there is no \textit{sof pasuq} (\Hebrew{׃}). This could possibly be used to argue the Philonic tradition.}%%
    \verse{20:4} \textsc{ii.} Never shalt thou make for yourselves graven images,\footnote{idols} neither any image that is in the heavens above, nor in the earth, nor beneath the earth, nor in the waters beneath the earth.%%
    \verse{20:5} And thou shalt not bow down to them, neither shalt thou worship them: for I, the \textsc{Lord} your God, am a jealous God and will seek retribution unto the third and fourth generation of them that hate me,%%
    \verse{20:6} but showing kindness\footnote{keeping my covenant} with\footnote{unto} those who love me\footnote{lit., my lovers} and\footnote{to those who} keep my commandments.%%
    \verse{20:7} \textsc{iii.} Never shalt thou use\footnote{take, lift up} the name of the \textsc{Lord} thy God with vain intent\footnote{in vain, with vanity, to/with no good purpose} for the \textsc{Lord} will not hold him innocent\footnote{guiltless} who uses His name with vain intent.\footnote{The real meaning here is to not take an oath in the name of God and not intend to keep it.}%%
    \verse{20:8} \textsc{iv.} Remember the Sabbath day to sanctify it.\footnote{consecrate, make it holy. The notion of \emph{making} the Sabbath day holy is more powerful than merely \emph{keeping} it holy for the responsibility then rests upon us to be an holy nation.}%%
    \verse{20:9} Six days shalt thou labor and do all thy work,%%
    \verse{20:10} but the seventh day, the Sabbath of the \textsc{Lord} thy God, never shalt thou do any work: neither thee, nor thy son, nor thy daughter, nor thy male or female servant,\footnote{lit., nor his/thy manservant, nor his/thy maidservant} nor thy beast, nor thy stranger that is within thy gates:%%
    \verse{20:11} for it took six days for the \textsc{Lord} to make the heavens and the earth and all that is upon the face thereof, and on the seventh day he rested. Therefore, the \textsc{Lord} blessed the Sabbath day and consecrated it.%%
    \verse{20:12} \textsc{v.} Take thy father and thy mother seriously\footnote{make their words heavy, honor them} so that thy days may be lengthened upon the land the \textsc{Lord} thy God giveth\footnote{Referring to the Promised Land that they have yet to inherit. It is rendered in the participle form thereby showing an ongoing action.} thee.%%
    \verse{20:13} \textsc{vi.} Never shalt thou murder.\footnote{It is not ``kill.'' The root that appears in the BHS (\Hebrew{רצח}) has behind it the idea of malicious forethought.}%%
    \verse{20:14} \textsc{vii.} Never shalt thou commit adultery.%%
    \verse{20:15} \textsc{viii.} Never shalt thou steal.%%
    \verse{20:16} \textsc{ix.} Never shalt thou answer falsely.\footnote{bear false witness/testimony}%%
    \verse{20:17} \textsc{x.} Never shalt thou desire\footnote{covet} thy neighbor's house, neither\footnote{\dots shalt thou desire\dots} thy neighbor's wife, nor his male or female servant, nor his ox, nor his male donkey, nor anything that is thy neighbors.''%%
    \verse{20:18} Then all the people were witnesses to the thunder,\footnote{lit., His voice} lightning, the sound of the trumpet, and the smoke of the mount. And they were witnesses and removed themselves.\footnote{In other words, they recognized the power and glory of God and stood back so as to not be consumed by His almighty power.}%%
    \verse{20:19} They then said to Moses, ``Speak on our behalf that we hear, and let Him not speak with us lest we die.''%%
    \verse{20:20} So Moses said unto the people, ``Do not be afraid,\footnote{Fear not} because in order to test thee, God is coming; and in order that thy reverence for Him be before you, that you don't sin.''%%
    \verse{20:21} The people stood back as Moses approached the thick cloud where God was.%%
    \verse{20:22} The \textsc{Lord} said to Moses, ``Thus shalt thou say unto the sons of Israel: `You have seen that I have spoken with you from the heavens.%%
    \verse{20:23} Never shalt thou make of me gods of gold or silver for yourselves.%%
    \verse{20:24} Thou shalt make for me an altar of earth and shalt offer unto me a burnt offering and a peace offering.%%
    \verse{20:25} But if you make an altar of stones to me, thou shalt not build it of hewn stones\footnote{Lest to be confused with an idol or graven image.} nor\footnote{lit., Never shalt thou} fashion those stones with tools: if thou wieldest thine tool\footnote{A metal instrument or tool. Not really a sword, although that is the word used in the BHS.} and lay it upon it\footnote{i.e., the altar} thou wilt defile it.%%
    \verse{20:26} Thou shalt not ascend on the steps to my altar in order that thy nakedness be not revealed on this altar.'\,''%%
\end{enumerate*}
