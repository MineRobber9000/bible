\heading{1}{xxxx}

\begin{inparaenum}
    \verse{1:1} %%
    \verse{1:2} %%
    \verse{1:3} %%
    
    \pvaa{\vn{1:4} The word of the \textsc{Lord} came to me, saying,}%%
    
    \pvab{\vn{1:5} ``Before I formed\footnotemark\ you in the womb, I knew you.}{Before you came out of the womb, I sanctified you.}%%
    \fnted{The verb used here, \Hebrew{יָצַר}, is defined in \textsc{halot} as generally meaning ``\textbf{form}, \textbf{shape}.'' However, when the subject is God, it means ``\textbf{create}, \textbf{form}'' but is noted as being an ``older, concrete synonym of \textit{b\=ar\=a'}''; since Jeremiah comes so late in the Hebrew period, it is doubtful, from a language standpoint, that the Lord is talking about creating. Additionally, from a theological and biological standpoint, it would be inconsistent to say that the Lord created Jeremiah in his mother's womb. This could only work poetically, not literally.}%%
    
    \pvaa{I made you a prophet to the nations.''\footnotemark}%%
    \fntca{\septuagint\superit{C}\super{a1} sg}{the Septuagint (textus Graecus in genere Catenarum traditus a1) has this in singular [i.e., ``to the nation'']}%%
    
    {\noindent\verse{1:6} I said, ``Ah, \textsc{Lord} God! I can't speak because I'm a child.''}%%
    
    \verse{1:7} %%
    
    \pvba{}%%
    
    \pvbb{}{}%%
    
    \pvab{\vn{1:8} }{}%%
    
    \pvab{}{\vn{1:9} }%%
    
    \pvbb{}{\vn{1:10} }%%
    
    \pvaa{}%%
    
    \verse{1:11} %%
    \verse{1:12} %%
    
    \verse{1:13} %%
    \verse{1:14} %%
    
    \pvbb{}{}%%
    
    \pvac{\vn{1:15} }{}{}%%
    
    \pvbb{}{}%%
    
    \pvbb{}{}%%
    
    \pvab{\vn{1:16} }{}%%
    
    \pvbb{}{}%%
    
    \pvbb{\vn{1:17} }{}%%
    
    \pvab{}{}%%
    
    \pvab{\vn{1:18} }{}%%
    
    \pvbb{}{}%%
    
    \pvbb{}{}%%
    
    \pvab{\vn{1:19} }{}%%
\end{inparaenum}
