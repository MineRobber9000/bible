\chapter{Covenants in Antiquity}\label{app:covenants-in-antiquity}
\section{\Hebrew{ידע}~--- to know}
% Split into enumerate
The verb \Hebrew{ידע} has three senses in BH. First, it means ``to know (a fact).''; for instance, ``To know the time.'' Second, it is used with a sexual connotation (to have sexual relations). Lastly, it is used in a covenantal sense~--- to enter a covenant (or treaty) with someone. Examples of this usage include:
\begin{itemize}
    \item ``And there arose a new king over Egypt, who did not \emph{know} Joseph'' (Exodus 1:8, \textsc{darby}, emphasis added). In other words, a king came to succession who had not covenanted with Joseph: ``I didn't know him, so all bets are off.''
    \item ``Before I formed thee in the belly I \emph{knew} thee'' (Jeremiah 1:4, \textsc{darby}, emphasis added).
    \item ``[A]nd then will I avow unto them, I never \emph{knew} you'' (Matthew 7:23, \textsc{darby}, emphasis added).
    \item ``\dots if thou art God, wilt thou make thyself known unto me, and I will give away all my sins to \emph{know} thee'' (Alma 22:18, emphasis added).
\end{itemize}

BH does not have a sense of \emph{conna\^\i tre}.\footnote{Fr. ``to know (a person)''} The closest to that sense is \Hebrew{נכר} which means, in the hiphil, \emph{to be acquainted}.

% This should be in a different section.
In ancient Israelite marriages, covenants were made to God, \emph{not} to the other person. Therefore, the breaker of the covenant must answer to God.

\section{Oath taking syntax}
The syntax of oath taking:

``I will not give your grain any longer as food for your enemies'' is literally ``\emph{If} I give your food to your enemies \emph{and} [understood: you will kill me].''

``If I don't do this, may my throat be slit just as the throat of this animal.''

cf.\ Alma 46:22--24. ``Preserved'' is a Muslim, not an Israelite, tradition.
