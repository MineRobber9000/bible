\section{Tabernacles and Temples}\label{app:tabernacle}
Numbers 10:11 contains the phrase \Hebrew{הָעֵדֻת מִשְׁכַּן} which is difficult to translate. The word \Hebrew{מִשְׁכַּן} is classically rendered \textit{tabernacle}, but Koehler-Baumgartner says the following:
\begin{quote}
    \textbf{dwelling-place, home} of Y.\footnote{meaning ``the \textsc{Lord}''}
\end{quote}
It can also mean tomb, sanctuary (especially the central sanctuary of Israel while in the desert), or tabernacle. The word \Hebrew{הָעֵדֻת} is classically rendered \textit{testimony}, but Koehler-Baumgartner says this about it:
\begin{quote}
    \textbf{warning signs, reminders, urgings}
\end{quote}
Making an idiomatic rendering of this proves difficult because the sense of ``the home of the \textsc{Lord}'' is important, but also stating that it is a home that is to serve as a reminder or an urging (most likely to be righteous).

In Numbers 12, the term \textit{tent} is interchangeable with \textit{tabernacle}. It is also this way in most of the Pentateuch.
