\chapter{Notes on Psalm 110}\label{app:psalm-110}
Psalm 110 deals with the endowment of a king in ancient Israel. Kings and prophets were in different orders of Priesthood, kings having a higher order. What is interesting about this is that the difference between becoming a king or a priest (the rites, at least) are not very different.

It is assumed that David wrote this psalm before he was crowned (i.e., when Saul was king). As a side note, one remarkable thing about David is that he was consistently obedient to the crown: he served the position, not the person.

The Aaronic, or Levitical, Priesthood was known anciently as \Hebrew{כְּהֻנָּה}.\footnote{from the root \Hebrew{כהנ}, meaning \emph{priest}} This priesthood was for Aaron and his descendants: ``And Aaron and his sons shalt thou appoint that they may attend to their priest's office'' (Numbers 3:10, \textsc{darby}). However, the higher priesthood, \Hebrew{דִּבְרָה},\footnote{As found in Psalm 110:4 as \Hebrew{מַלְכִּי־צֶֽדֶק עַל־דִּבְרָתִי}} was given to prophets and kings. The assumption is that all who reigned in Israel had this latter order of priesthood.

This higher authority allowed those in its possession to enter the Holy Place and the Holy of Holies without particular regard to worthiness (as compared to those of the Aaronic order who had to be ritualistically and ethically clean among other prerequisites). However, an interesting story is found in 2~Chronicles chapter~26 where king Uzziah (the ruler at the time of Isaiah) assumed that he had this authority and walked into the Holy of Holies\footnote{While it is not explicitly stated in 2~Chronicles~26 that he entered the Holy of Holies, it is stated that he went to burn incense before the Lord~--- something that was done in the Holy of Holies.} and got leprosy.\footnote{Most likely some skin disease (\Hebrew{צְרוּעָ}) and not necessarily leprosy} One possible explanation for this is that Uzziah was king of Judah, not Israel, and this may not have been sufficient for him to be of the higher order of priesthood.
