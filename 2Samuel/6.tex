\heading{6}{King David gathers the people together~--- they take the Ark from Abinadab's house to Obed-Edom's~--- the cart carrying the Ark rocks, Uzzah tries to steady it and is struck dead~--- David rejoices and dances about half-naked~--- Saul's daughter Michal reproves him~--- David responds like a jerk}

\begin{inparaenum}
    \verse{6:1} David once again gathered every firstborn in Israel, 30\thinspace000.%%
    \verse{6:2} David and all the people that were with him got up and went from Baale-Judah\footnote{Known as Kiriath-jearim in \vref{1~Chronicles}{13}{6}.} to take up the Ark of God. God,\understood\ whose name has been invoked upon it~--- that name being ``The \textsc{Lord} of Hosts who sits between the cherubs.''%%
    \verse{6:3} The Ark of God rode on a new wagon.\footnote{KB: not a chariot.} They lifted it up from Abinadab's house in Gibeah. Uzzah and Ahio, Abinadab's sons, led the new wagon.%%
    \verse{6:4} They carried the Ark of God from Abinadab's house on the hill, Ahio going before the Ark.%%
    \verse{6:5} David and the Israelites played \footnote{lit., with}all kinds of instruments\understood\ before the \textsc{Lord}, instruments\understood\ of juniper wood,\footnote{KB: Phoenician juniper, \textit{Juniperus phoenicea} (tree \& wood).} lyres,\footnote{alt., harp} harps, drums, sistrums,\footnote{KB: small percussion instrument which is rattled.} and cymbals.%%
    \verse{6:6} They came to Nachon's threshing floor and Uzzah put forth his hand\understood\ to the ark of God and took hold of it because the oxen stumbled.%% Reword so the oxen stumbling is earlier in the sentence.
    \verse{6:7} And the \textsc{Lord} was exceedingly wroth with Uzzah, so God smote him there for his error that he died there before the ark of God.\footnote{John Taylor said, referring to verses 6--7, ``The ark of God does not need steadying, especially by incompetent men without revelation and without knowledge of the kingdom of God and its laws.'' (\textit{The Gospel Kingdom}, 166)}%%
    \verse{6:8} It was angering to David that the \textsc{Lord} had torn a breach through Uzzah. This place, to this day, is called Perez-Uzzah (the breach of Uzzah).\footnote{Translation provided in-line.}%%
    \verse{6:9} That day, David reverenced the \textsc{Lord} and said, ``Why should the Ark of the \textsc{Lord} come to me?''%%
    \verse{6:10} But David himself was not willing to remove the Ark of the \textsc{Lord} to the City of David; rather, he\lit{David} turned aside to the Gittite\footnote{Someone from Gath (where Goliath was from).} Obed-Edom's house%%
    \verse{6:11} and the Ark of the \textsc{Lord} stayed there,\understood\ in the house of the Gittite Obed-Edom, for three months. And the \textsc{Lord} blessed Obed-Edom and his family.%%
    \verse{6:12} This was explained to King David, saying, ``The \textsc{Lord} blessed Obed-Edom's family and everything that he has because of the Ark of God.'' So David went and joyfully\footnote{lit., with joy} brought up the Ark of God from Obed-Edom's house to the City of David.%%
    \verse{6:13} When those who were bearing the Ark of the \textsc{Lord} had walked six steps, he sacrificed a ox and a fattened cattle.%%
    \verse{6:14} David, dressed in a linen ephod, danced before the \textsc{Lord} with all his might.%%
    \verse{6:15} David and all the people of Israel brought up the Ark of the \textsc{Lord} while shouting and playing the shofar.\footnote{lit., with shouting and the voice/sound of the shofar.}%%
    \verse{6:16} The Ark of the \textsc{Lord} came into the City of David. Michal, Saul's daughter, looked down from her window and saw King David being nimble and dancing before the \textsc{Lord}. And she thought contemptuously of him in her heart.%%
    \verse{6:17} They brought the Ark of the \textsc{Lord} and placed it in its spot\footnote{alt., set it in its place} in the tent that David had set up. And David offered burnt-offerings and peace-offerings before the \textsc{Lord}.%%
    \verse{6:18} When David completed offering the burnt-offering and the peace-offering, he blessed the people in the name of the \textsc{Lord} of Hosts%%
    \verse{6:19} and allotted to all of the people~--- the whole crowd of Israel, men and women~--- to each he allotted\understood\ a ring-shaped loaf\understood\ of bread, a date-cake, a raisin-cake. And everyone went to their homes.%%
    \verse{6:20} David returned to bless his house and Saul's daughter, Michal, went out to call on David, and said, ``How magnificent\footnote{xxxx: check if we can justify this rendering instead of ``honorable.''} was the king of Israel today that he exposed himself today in front of\footnote{lit., in the eyes of} his servants' handmaids, just how\footnote{alt., like, as} an uncovered, vain person is exposed.''%%
    \verse{6:21} David said to Michal, ``It was before the \textsc{Lord} (who chose me instead of your dad,\footnote{Jerk comment. Granted she wasn't being very polite, but this is no way to respond.} instead of his whole household, and appointed me leader over the people of \textsc{God}, the Israelites),\footnote{lit., over the people of Israel} so I danced before the \textsc{Lord}.%%
    \verse{6:22} I have been more humble\footnote{This is a difficult verb to render. \Hebrew{קלל} appears here in the Niphal, \Hebrew{וּנְקַלֹּתִי}, and means: %%
    \begin{inparaenum}
        \item \textbf{prove swift}
        \item \textbf{humble oneself, demean oneself}
        \item be a small matter to someone
        \item \textbf{be easy to \dots}
        \item \textbf{be too light a thing to \dots}
        \item \textbf{be easy; superficially}
    \end{inparaenum}%%
    . It seems that David is saying that he is humbling himself, yet \textsc{darby} renders it as ``I will make myself yet more vile than thus'' and \textsc{ylt} renders it as ``I have been more vile than this.'' In essence, however, \Hebrew{קלל} means ``to humble oneself.''} than this. In my opinion, I've been lower.\footnote{KB: in social respect} I will be honored along with the handmaids you've referred to.''\footnote{lit., spoken of}%%
    \verse{6:23} \footnote{lit., to\dots}Michal, Saul's daughter, didn't have a child until the day of her death.\footnote{How is this relevant?}\footnote{Does this mean that she never had a child, or that she had a child and died that selfsame day?}%%
\end{inparaenum}
